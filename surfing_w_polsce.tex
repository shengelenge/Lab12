\documentclass{article}

\usepackage{polski}
\usepackage{lipsum}
\usepackage[margin=1in, left=1.5in, includefoot]{geometry}

% Header and Footer
\usepackage {fancyhdr}
\pagestyle{fancy}
\fancyhead{}
\fancyfoot{}
\fancyfoot[R]{ \thepage\ }
\renewcommand{\headrulewidth}{0 pt}
\renewcommand{\footrulewidth}{1 pt}


\begin{document}

\begin{titlepage}
	\begin{center}
	\line(1,0){300} \\
	[0.25in]
	\huge{\bfseries Surfing w Polsce} \\
	[2mm]
	\line(1,0){200} \\
	[1.5cm]
	\textsc{\LARGE Uniwersytet Gdański} \\
	[0.75cm]
	\textsc{\Large Projekt o surfingu w Polsce} \\
	[10cm]
	\end{center}
	\begin{flushright}
	\textsc{\large Nicolas W. \\
	Środowisko Programisty \\ 
	\# 275512 \\
	8 stycznia 2021 \\}
	\end{flushright}
	
	
\end{titlepage} 
\tableofcontents
\thispagestyle{empty}
\cleardoublepage

\setcounter{page}{1}



\section{Wstęp}\label{sec:intro} 
Surfing kojarzony jest głównie z oceanem, plażami i słońcem. Większość ludzi w Polsce  nie wie na czym polega ten sport, lub nie miało pojęcia, że można go uprawiać w Polsce. \\
Dla tych co nigdy nie słyszeli o surfingu, jest to sport wodny, który polega na przemieszczaniu po łamiącej się fali, głównie przy pomocy deski surfingowej.


Korzenie surfingu sięgają XIX-wiecznych Hawai oraz Polinezji Francuskiej gdzie sport był uprawiany przez wszystkich ludzi z różnych klas społecznych. Dzięki turystyce sport szybko rozprzestrzenił się w Kalifornii oraz Australii. Już w latach sześćdziesiątych XX wieku surfing był znany na całym świecie oraz miały miejsce pierwsze zawody surfingowe.


Również Polacy znają tę dyscyplinę od wielu lat, jednak dopiero od niedawna dowiedzieliśmy się, że Bałtyk produkuje fale dorównujące tym z nad oceanu. Rożnica jest taka, że u nas surfing nie kojarzy się z ciepłem i słońcem.

\newpage
\section{Warunki}
Surfing różni się od  sportów typu piłka nożna czy rugby, żeby móc płynąć po fali potrzebne są odpowiednie warunki pogodowe zarówno w czasie uprawiania sportu jak i przed. 
\subsection{Burze}
To właśnie burze i sztormy na środku oceanu tworzą fale, które przez tysiące kilometrów wędrują przez ocean, by na końcu rozbić się na wybrzeżu.
\subsection{Wiatr}
Wiatr jest głównym składnikiem powstawania fal na Bałtyku oraz na wielu innych morzach.
\subsubsection{Odpowiednie kierunki wiatru}
Na naszym morzu można pływać  przy kierunkach:
	\begin{itemize}	
	\item północnych (N)
	\item północno-zachodnich (NW)
	\item północno-wschodnich (NE)
	\item zachodnich (W)
	\item wschodnich (E)
	\end{itemize}
Oczywiście jest to spowodowane geograficznym położeniem naszego wybrzeża.
\subsection{Podłoże}
Prócz optymalnych warunków pogodowych potrzebnych do wytworzenia się fali, także należy zwrócić uwagę na podłożę pod wodą oraz otaczający krajobraz.  \\
W skrócie, gdy jest głęboko fala jest płaska i z reguły wolniejsza, a płycizna powoduje, że fala robi się stroma i szybciej się załamuje przez co ma więcej energii.

\end{document}